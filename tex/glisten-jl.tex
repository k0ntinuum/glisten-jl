

\documentclass{article}
\usepackage[utf8]{inputenc}
\usepackage{setspace}
\usepackage{ mathrsfs }
\usepackage{amssymb} %maths
\usepackage{amsmath} %maths
\usepackage[margin=0.2in]{geometry}
\usepackage{graphicx}
\usepackage{ulem}
\setlength{\parindent}{0pt}
\setlength{\parskip}{10pt}
\usepackage{hyperref}
\usepackage[autostyle]{csquotes}

\usepackage{cancel}
\renewcommand{\i}{\textit}
\renewcommand{\b}{\textbf}
\newcommand{\q}{\enquote}
%\vskip1.0in



\begin{document}

{\setstretch{0.0}{


\b{Cyclone} is a symmetric cryptosystem inspired loosely by the famous Enigma machine. The $n \times n$ square matrix key is best thought of as a cylinder of stacked wheels.

Each row or a \q{wheel} of the key is a permutation on $\{1,2,3,...,n\}$ expressed in the abbreviated form of  $ f(1)f(2) ... f(n-1)f(n) $. 

Let $f_i$ be the permutation of row $i$. Then $f = f_1 \circ f_2 \circ ... f_{n-1} \circ f_{n}$ is itself a permutation, and it is this $f$ which is used to transform a symbol of the plaintext into a symbol of the ciphertext. Note that this means $f$ is invertible, which is necessary for decoding. 

Then this $f$ is shifted to the right in a circular fashion by the value of the plaintext symbol. Finally, each row of the key is itself shifted using this shifted $f$. 

Here's the crucial functions for encoding a single round: 

\begin{verbatim}
function get_f(k)
    f = Int64[]
    n = size(k)[begin]
    for i in 1:n
        x = i
        for j in 1:n x = k[j,x ] end
        push!(f, x)
    end
    f
end


function encode(p,q,F)
    k = copy(q)
    c = Int64[]
    s = zeros(Int64,n)
    for i in eachindex(p)
        f = get_f(k)
        push!(c,f[p[i]] )
        #c[i] = f[p[i]]
        f = circshift(f,p[i])
        push!(F,f)
        for j in 1:n k[j,:] = circshift(k[j,:], f[j]) end
    end
    c
end

\end{verbatim}

This is the essence of the algorithm, but the encrypt function uses features like autospin and reversal to make an arbitrary number of rounds (hopefully) more effective. 

When $n = 27$, the compositions $f$ tend to be unique. So the key constantly wanders into a new state, meeting each symbol of plaintext with a fresh permutation. I also created a system called Loop which is like a miniature version of Cyclone, using just one row as the entire key. 


}}
\end{document}
